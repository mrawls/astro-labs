\documentclass[notitlepage]{article}
\newcounter{qcounter}
\usepackage{graphicx}
\usepackage{array}
\usepackage{booktabs}
\usepackage{fancyhdr}
\usepackage{amsmath}
\usepackage{url}
\pagestyle{empty}
\textwidth	7.00in
\textheight	8.50in
\oddsidemargin	-0.35in
\topmargin	-0.5in
\parindent	0.00in




\begin{document}
 
 \begin{center}
  \Large{\bf Lab 10 }\\
  \Large{\bf The Power of Light: Understanding Spectroscopy}\\
  \vspace{0.5cm}
  Names: \vbox{\hrule width 10cm}
 \end{center}

\large{\bf 10.5 Observing Blackbody Sources with the Spectrograph}\\

3. \textbf{Brighter (hotter) setting}: Do you see light at all different wavelengths/colors or only a few discrete wavelengths? (2pts)
\vspace{3cm}

4. Of all the colors which you see in the spectrograph, which color appears the brightest? (3pts)
\vspace{2cm}

5. \textbf{Dimmer (cooler) setting}: Do you see light at all different wavelengths/colors or only a few discrete wavelengths? Of all the colors which you see in the spectrograph, which color appears the brightest? (3pts)
\vspace{3cm}


6. Describe the changes between the two light bulb observations. What happened to the spectrum as the brightness and temperature of the light bulb increased?  Specifically, \textbf{what happened to the relative amount of light at different wavelengths}? (5pts)
\vspace{3.5cm}

7. Betelgeuse is a Red Giant star found in the constellation Orion. Sirius, the brightest star in the sky, is much hotter and brighter than Betelgeuse.  Describe how you might expect the colors of these two stars to differ. (4pts)
\vspace{3cm}




\large{\bf 10.6 Quantitative Behavior of Blackbody Radiation}\\

Website to use: \url{http://astro.unl.edu/naap/blackbody/animations/blackbody.html}\\

1. Set the temperature to 6000 K. Note the wavelength, and the color of the spectrum at the peak of the blackbody curve. 
\vspace{2cm}


2. Set the temperature to 3000 K. How do the spectra differ? Consider both the \textit{relative} amount of light at different wavelengths as well as the overall \textit{brightness}. Now set the temperature to 12,000 K. How do the spectra differ? (5pts)
\vspace{4cm}

3. Fill in the table of peak wavelengths for the following temperatures. (5 pts)

\begin{table}[h]
\centering
\caption{{(\bf 5 points})}
\begin{tabular}{|c|c|}
\hline
Temperature $T$ (K) & Peak Wavelength $\lambda$ (nm) \\[0.2cm]
\hline
3000 & \\[0.2cm]
\hline
6000 & \\[0.2cm]
\hline
12000 & \\[0.2cm]
\hline
24000 & \\[0.2cm]
\hline

\end{tabular}
\end{table}

4. Can you see a pattern from your table?  Given the mathematical expression
$T  = \frac{C}{\lambda}$,
plug in some numbers to solve for the constant $C$ (don't forget units!).  ({\bf 3 points})
\vspace{3cm}



5. Where do you think the peak wavelength would be for objects on Earth, at $T =$ 300 K? (2 pts)

\vspace{2cm}


\large{\bf 10.7 Spectral Lines Experiment}\\

On the attached graphs, make a drawing of the lines you see.  Label each Element. (12 pts)
\vspace{1cm}


\large\textbf{10.7.3 The Unknown Element}\\

Write down the wavelengths of the spectral lines that you can see in the table below, and note their color:



\begin{table}[h]
\centering
\begin{tabular}{|c|c|}
\hline
Observed Wavelength (nm) & Color of line \\[0.2cm]
\hline
 & \\[0.2cm]
\hline
 & \\[0.2cm]
\hline
 & \\[0.2cm]
\hline
 & \\[0.2cm]
\hline
 & \\[0.2cm]
\hline
 & \\[0.2cm]
\hline

\end{tabular}
\end{table}


Which element is the unknown element? Explain how you know. (5 pts)

\vspace{5cm}

\large{\bf 10.8 Questions}
\begin{enumerate}
\item Describe in detail why the emission or absorption from a particular electron would produce lines only at specific wavelengths rather than at all wavelengths like a blackbody. (Hint: use the Bohr model to help you answer this question.) (5 pts)

\vspace{7cm}

\item What causes a spectrum to have more lines than another spectrum (for example, Helium has more lines than Hydrogen)? (4 pts)

\vspace{5cm}

\item Referring to Fig. 10.3, does the electron transition in the atom labeled ``A'' cause the emission of light, or require the absorption of light? (2 pts)

\vspace{3cm}

\item Referring to Fig. 10.3, does the electron transition in the atom labeled ``B'' cause the emission of light, or require the absorption of light? (2 pts)
 
\vspace{3cm}

\item Comparing the atom labeled ``C'' to the atom labeled ``D'', which transition (that occurring in C, or D) releases the largest amount of energy? (3 pts)


%\begin{figure}
%\centering
%\includegraphics[width=5cm]{./atom_diagram.png}
%\label{ad}
%\caption{Electron transitions in an atom (the electrons are the small dots, the nucleus the large black dots, and the circles are possible orbits.}
%\end{figure}

\end{enumerate}



\end{document}
