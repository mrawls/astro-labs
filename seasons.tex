\documentclass[notitlepage]{article}
\newcounter{qcounter}
\usepackage{graphicx}
\usepackage{array}
\usepackage{booktabs}
\usepackage{fancyhdr}
\pagestyle{empty}
%\pagestyle{fancy}
%\fancyhf{}
%\fancyfoot[HR]{Page \thepage\ }
%\renewcommand{\headrulewidth}{0pt}
%\renewcommand{\footrulewidth}{0pt}
\textwidth	7.00in
\textheight	8.50in
\oddsidemargin	-0.35in
\topmargin	-0.5in
\parindent	0.00in
%\newcolumntype{m}{{\centering\arraybackslash}m{\dimexpr.25\linewidth-2\tabcolsep}}

\begin{document}

\begin{center}
\Large{\bf Lab 2}\\
\Large{\bf The Origin of the Seasons}\\
\vspace{0.5cm}
Names: \vbox{\hrule width 10cm}
\end{center}
%\vspace{1cm}

%\begin{list}{\bf Exercise \arabic{qcounter}:~}{\usecounter{qcounter}}

\large{\bf 2.2 The Seasons}\\
\begin{enumerate}
\item Do you think this change in distance is big enough to cause the seasons? Explain your logic. (3pts) \\
\vspace{2.0cm}

\item Take the ratio of the aphelion distance / perihelion distance: \vbox{\hrule width 4cm}
\\\\\\
Sun diameter in January image = \vbox{\hrule width 4cm} mm\\\\
Sun diameter in July image = \vbox{\hrule width 4cm} mm\\\\

\item Take the ratio of bigger diameter / smaller diameter: \vbox{\hrule width 4cm}. (1pt) 
\\\\

\item How does this ratio compare to the ratio you calculated in question \#2? (2pts) \\
\vspace{1.5cm}

\item If an object appears bigger when we get closer to it, when is the Earth closest to the Sun? (2pts) \\
\vspace{0.5cm}

\item At that time of year, what season is it in Las Cruces? What do you conclude about the statement, ``the seasons are caused by the changing distance between the Earth and the Sun''? (4pts) \\
\vspace{2cm}

\item Thus, for Las Cruces, the Sun is ``up'' longer in July than in January. Is the same thing true for all cities with northern latitudes? \vbox{\hrule width 2cm} (1pt) 
\\\\
\item Fairbanks is \vbox{\hrule width 2cm} the North Pole than Las Cruces. (1pt) \\
\\\\
\item In January, there are more daylight hours in \vbox{\hrule width2cm} (1pt) \\
\\\\
\item In July, there are more daylight hours in \vbox{\hrule width 2cm} (1pt) \\
\\\\
\item While the latitudes of Las Cruces and Sydney are similar, Las Cruces is \vbox{\hrule width 2cm} of the Equator, and Sydney is \vbox{\hrule width 2cm} of the Equator. (2pts) \\
\\\\
\item In January, there are more daylight hours in \vbox{\hrule width 2cm}. (1pt)\\
\\\\
\item In July, there are more daylight hours in \vbox{\hrule width 2cm}. (1pt) \\
\\\\
\item Summarizing: During the Wintertime (January) in both Las Cruces and Fair banks there are fewer daylight hours, and it is colder. During July, it is warmer in both Fairbanks and Las Cruces, and there are more daylight hours. Is this also true for Sydney?: \vbox{\hrule width 2cm} (1pt) \\
\\
\item In fact, it is wintertime in Sydney during \vbox{\hrule width 2cm} and summertime during \vbox{\hrule width 2cm} (2pt) \\ \\

\item From Table 2.1, I conclude that the times of the seasons in the Northern hemisphere are exactly \vbox{\hrule width 3cm} to those in the Southern hemisphere. (1 pt) 

\large{\bf The Spinning, Revolving Earth}\\
{\bf Experiment \#1: Equinox}\\

Record the full circumference of each circle here\\\\
Equator \vbox{\hrule width 2.2cm}, 45$^{\circ}$ N \vbox{\hrule width 2.2cm}, Arctic \vbox{\hrule width 2.2cm}, Antarctic \vbox{\hrule width 2.2cm}\\

\begin{center}
Table 2.2: Position \#1: Equinox Data Table

\begin{table}[h]
\centering
\begin{tabular}{| >{\centering\arraybackslash} m{2cm} | >{\centering\arraybackslash} m{5cm} | >{\centering\arraybackslash} m{5cm} | }
\hline
{\bf Latitude} & {\bf Length of Daylight Arc} & {\bf Length of Nighttime Arc}\\
\hline
Equator & & \\[0.5cm]
\hline
45$^{\circ}$ N & & \\[0.5cm]
\hline
Arctic Circle & & \\[0.5cm]
\hline
Antarctic Circle & & \\[0.5cm]
\hline
\end{tabular}
\end{table}
\vspace{1cm}
Table 2.3: Position \#1: Equinox Length of Day and Night

\begin{table}[h]
\centering
\begin{tabular}{| >{\centering\arraybackslash} m{2cm} | >{\centering\arraybackslash} m{5cm} | >{\centering\arraybackslash} m{5cm} | }
\hline
{\bf Latitude} & {\bf Daylight Hours} & {\bf Nighttime Hours}\\
\hline
Equator & & \\[0.5cm]
\hline
45$^{\circ}$ N & & \\[0.5cm]
\hline
Arctic Circle & & \\[0.5cm]
\hline
Antarctic Circle & & \\[0.5cm]
\hline
\end{tabular}
\end{table}

\end{center}

\vspace{0.5cm}

\item The caption for Table 2.2 was ``Equinox data''. The word Equinox means ``equal nights'', as the length of the nighttime is the same as the daytime. While your numbers in Table 2.3 may not be exactly perfect, what do you conclude about the length of the nights and days for all latitudes on Earth in this experiment? Is this result consistent with the term Equinox? (3pts) \\
\vspace{3cm}
\newpage

{\bf Experiment \#2: Solstice 1}\\
\begin{center}
Table 2.4: Position \#2: Solstice 1 Data Table
\begin{table}[h]
\centering
\begin{tabular}{| >{\centering\arraybackslash} m{2cm} | >{\centering\arraybackslash} m{5cm} | >{\centering\arraybackslash} m{5cm} | }
\hline
{\bf Latitude} & {\bf Length of Daylight Arc} & {\bf Length of Nighttime Arc}\\
\hline
Equator & & \\[0.5cm]
\hline
45$^{\circ}$ N & & \\[0.5cm]
\hline
Arctic Circle & & \\[0.5cm]
\hline
Antarctic Circle & & \\[0.5cm]
\hline
\end{tabular}
\end{table}

\vspace{1cm}

Table 2.5: Position \#2: Solstice 1 Length of Day and Night

\begin{table}[h]
\centering
\begin{tabular}{| >{\centering\arraybackslash} m{2cm} | >{\centering\arraybackslash} m{5cm} | >{\centering\arraybackslash} m{5cm} | }
\hline
{\bf Latitude} & {\bf Daylight Hours} & {\bf Nighttime Hours}\\
\hline
Equator & & \\[0.5cm]
\hline
45$^{\circ}$ N & & \\[0.5cm]
\hline
Arctic Circle & & \\[0.5cm]
\hline
Antarctic Circle & & \\[0.5cm]
\hline
\end{tabular}
\end{table}

\end{center}

\vspace{1cm}

\item Compare your results in Table 2.5 for +45$^{\circ}$ latitude with those for Minneapolis in Table 2.1. Since Minneapolis is at a latitude of +45$^{\circ}$ , what season does this orientation of the globe correspond to? (2 pts) \\
\vspace{2cm}

\item What about near the poles? In this orientation what is the length of the nighttime at the North pole, and what is the length of the daytime at the South pole? Is this consistent with the trends in Table 2.1, such as what is happening at Fairbanks or in Ushuaia? (4 pts) \\
\vspace{3cm}
\newpage

{\bf Experiment \#3: Solstice 2}\\
\begin{center}
Table 2.6: Position \#3: Solstice 2 Data Table
\begin{table}[h]
\centering
\begin{tabular}{| >{\centering\arraybackslash} m{2cm} | >{\centering\arraybackslash} m{5cm} | >{\centering\arraybackslash} m{5cm} | }
\hline
{\bf Latitude} & {\bf Length of Daylight Arc} & {\bf Length of Nighttime Arc}\\
\hline
Equator & & \\[0.5cm]
\hline
45$^{\circ}$ N & & \\[0.5cm]
\hline
Arctic Circle & & \\[0.5cm]
\hline
Antarctic Circle & & \\[0.5cm]
\hline
\end{tabular}
\end{table}

Table 2.7: Position \#3: Solstice 2 Length of Day and Night
\begin{table}[h]
\centering
\begin{tabular}{| >{\centering\arraybackslash} m{2cm} | >{\centering\arraybackslash} m{5cm} | >{\centering\arraybackslash} m{5cm} | }
\hline
{\bf Latitude} & {\bf Daylight Hours} & {\bf Nighttime Hours}\\
\hline
Equator & & \\[0.5cm]
\hline
45$^{\circ}$ N & & \\[0.5cm]
\hline
Arctic Circle & & \\[0.5cm]
\hline
Antarctic Circle & & \\[0.5cm]
\hline
\end{tabular}
\end{table}

\end{center}


\item Compare the results found here for the length of daytime and nighttime for the +45$^{\circ}$ degree latitude with that for Minneapolis. What season does this appear to be? (2 pts) \\
\vspace{1cm}

\item What about near the poles? In this orientation, how long is the daylight at the North pole, and what is the length of the nighttime at the South pole? Is this consistent with the trends in Table 2.1, such as what is happening at Fairbanks or in Ushuaia? (2 pts) \\
\vspace{2cm}

\item Using your results for all three positions, can you explain what is happening at the Equator? Does the data for Quito in Table 2.1 make sense? Why? Explain. (3 pts) \\
\newpage

{\bf 2.4 Elevation Angle and the Concentration of Sunlight}\\\\
Turn on the flashlight and move the arm to lower and higher angles. How does the illumination pattern change? Does the illuminated pattern appear to change in brightness as you change angles? Explain. (2 points) \\
\vspace{3cm}

The diameter of the illuminated circle is \vbox{\hrule width 2cm} cm. \\\\

The area of the circle of light at an elevation angle of 90$^{\circ}$ is \vbox{\hrule width 2cm} cm$^2$. (1 pt) \\\\

The major axis has a length of a = \vbox{\hrule width 2cm} cm, while the minor axis has a length of b = \vbox{\hrule width 2cm} cm. \\\\


The area of an ellipse is simply ($\pi$ $\times$ a $\times$ b)/4. So, the area of the ellipse at an elevation angle of 45$^{\circ}$: \vbox{\hrule width 2cm} cm$^2$ (1 pt). \\\\


At 90$^{\circ}$, the amount of light per centimeter is 100 divided by the area of circle = \vbox{\hrule width 2cm} units of light per cm$^2$ (1 pt). \\\\


At 45$^{\circ}$, the amount of light per centimeter is 100 divided by the area of the ellipse = \vbox{\hrule width 2cm} units of light per cm$^2$ (1 pt). \\\\
%\vspace{2cm}

Since light is a form of energy, at which elevation angle is there more energy per square centimeter? Since the Sun is our source of light, what happens when the Sun is higher in the sky? Is its energy more concentrated, or less concentrated? How about when it is low in the sky? Can you tell this by looking at how bright the ellipse appears versus the circle? (4 pts) \\\\
\vspace{4cm}





\end{enumerate}
%\end{list}


\end{document}