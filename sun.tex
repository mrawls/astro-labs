\documentclass[notitlepage]{article}
\newcounter{qcounter}
\usepackage{graphicx}
\usepackage{array}
\usepackage{booktabs}
\usepackage{fancyhdr}
\pagestyle{empty}
\textwidth	7.00in
\textheight	8.50in
\oddsidemargin	-0.35in
\topmargin	-0.5in
\parindent	0.00in

\begin{document}
 
 \begin{center}
  \Large{\bf Lab 6}\\
  \Large{\bf Our Sun}\\
  \vspace{0.5cm}
  Names: \vbox{\hrule width 10cm}
 \end{center}

 
\large{\bf Exercise \#1}
{\bf Station 2}
\begin{itemize}
  \item SOHO Images  ({\it 8 points})
  \begin{itemize}
    \item If the Sun turns by 90 degress in a time $t$, it would complete one revolution of 360 degrees in how much time?
    \vspace{4cm}
    \item Does this match the rotation rate given in your textbook or in lecture? Cite a source for this value. 
    \vspace{4cm}
  \end{itemize}
  \item Magnetogram ({\it 7 points})
  \begin{itemize}
    \item What do you notice about the location of {\it sunspots} in the photo and the location of the {\it strongest magnetic fields}, shown by the brightest or darkest colors in the magnetogram?
    \vspace{4cm}	
    \item Based on this answer, what do you think causes sunspots to form? Why are they dark?
    \vspace{4cm}
  \end{itemize}
\end{itemize}
{\bf Station 3} ({\it 10 points})
\begin{itemize}
 \item Diameter of the Sun (on paper): \vbox{\hrule width 8cm}
 
    \vspace{1cm}
    Minimum extent of the corona (on paper): \vbox{\hrule width 8cm}
    
    \vspace{1cm}

    \item Size of the corona (in reality): \vbox{\hrule width 8cm}
 
    \vspace{1cm}
    \item How many times larger than the Earth is the corona? \vbox{\hrule width 8cm}
 
    \vspace{1cm}
\end{itemize}

 {\bf Station 4} ({\it 15 points})
 \begin{itemize}
  \item Diameter of the Sun (on paper): \vbox{\hrule width 8cm}
  
  \vspace{1cm}
    First prominence distance (on paper): \vbox{\hrule width 8cm}
   
   \vspace{1cm}
    Second prominence distance (on paper): \vbox{\hrule width 8cm}
    
   \vspace{1cm}
    How far did the prominence move (on paper)? \vbox{\hrule width 8cm}
  
  \vspace{1cm}
  \item Diameter of the Sun (in reality): \vbox{\hrule width 8cm}
  
    \vspace{1cm}
    How far did the prominence move (in reality)? \vbox{\hrule width 8cm}

    
    \vspace{1cm}
  \item Velocity of the prominence: \vbox{\hrule width 8cm}
    \vspace{1cm}
  \item Time to reach the Earth if a prominence were moving at 2000 km/s: 
    \vspace{1cm}
 \end{itemize}

 {\bf Station 5} 
 What do you notice about the distribution of sunspots? How long does it take for the pattern to repeat? What does the length of time correspond to? ({\it 3 points})
 \vspace{8cm}
 
 
 \large{\bf Exercise \#2}
 \begin{itemize}
  \item Which end of the compass needle (or arrow) seems to be attracted by the north pole of the magnet? ({\it 1 point})
  \vspace{2cm}
  \item Which end of the compass needle is attracted to the south pole of the bar magnet? ({\it 1 point})
  \vspace{2cm}
  \item Which pole is attracted to which pole? ({\it 1 point})
  \vspace{2cm}
  \item What is the actual ``polarity'' of the Earth's ``magnetic North'' pole?
  \vspace{2cm}
  \item Sketch the pattern traced out by the magnetic filings below and \textbf{describe} the pattern in words. ({\it 2 points})
  \vspace{7cm}
  \item What does this imply about sunspots? ({\it 2 points})
  \vspace{3cm}
  \item Draw the field lines \textbf{above} a bar magnet. ({\it 2 points})
  \vspace{7cm}
  \item What are the similarities between your drawing and the loop prominence from station \#1 of Exercise \#1? ({\it 1 point})
  \vspace{3cm}
 \end{itemize}
 
 
% \large{\bf Exercise \#3}
% \begin{itemize}
%  \item On the ``Solar Observation Worksheet'', draw what you see on and near the Sun as seen through the special solar telescope. ({\it 8 points})
% \end{itemize}


 
 
 
 
 
 
 
\end{document}




