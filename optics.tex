\documentclass[notitlepage]{article}
\newcounter{qcounter}
\usepackage{graphicx}
\usepackage{array}
\usepackage{booktabs}
\usepackage{fancyhdr}
%\pagestyle{empty}
\textwidth	7.00in
\textheight	8.50in
\oddsidemargin	-0.35in
\topmargin	-0.5in
\parindent	0.00in

\begin{document}
 
 \begin{center}
  \Large{\bf Lab 9: Optics}\\
  \vspace{0.5cm}
  Names: \vbox{\hrule width 10cm}
 \end{center}
 \vspace{0.5cm}

\large{\bf 9.3 Reflective Optics: Mirrors}\\

Make a sketch of what you observe on the attached worksheet (Fig. 9.4).
%\vspace{6cm}

\begin{table}[h]
\centering
\caption{{\bf (3 points)}}
\begin{tabular}{|c|c|}
\hline
Angle of Incidence & Angle of Reflection \\[0.2cm]
\hline
20$^{\circ}$ & \\[0.2cm]
\hline
30$^{\circ}$ & \\[0.2cm]
\hline
45$^{\circ}$ & \\[0.2cm]
\hline
60$^{\circ}$ & \\[0.2cm]
\hline
75$^{\circ}$ & \\[0.2cm]
\hline
90$^{\circ}$ & \\[0.2cm]
\hline
\end{tabular}
\end{table}

What do you conclude about how light is reflected from a mirror? ({\bf 2 points})
\vspace{3cm}

Draw your prediction for a curved mirror in the space below. 

\vspace{5cm}

Was your prediction correct? Explain.
%\vspace{2cm}
\newpage

Explain in words what happens with the \textit{concave} mirror, and draw a diagram on the attached worksheet (Fig. 9.4). ({\bf 5 points})
\vspace{3.5cm}

Explain in words what happens with the \textit{convex} mirror, and draw a diagram on the attached worksheet (Fig. 9.4). ({\bf 5 points})
\vspace{3.5cm}

Which of the reflected beams disappeared when the top laser was blocked? Sketch what happened below. ({\bf 5 points})
\vspace{4cm}

Where is the best focus achieved for the big concave mirror (\textit{shared class result})? ({\bf 3 points})
\vspace{2cm}

What is the radius of curvature of the big concave mirror? ({\bf 1 point})
\vspace{2cm}

Is your face larger or smaller? Does a \textit{concave} mirror magnify or demagnify? Does a \textit{convex} mirror magnify or demagnify? ({\bf 1 point})
\vspace{4cm}



\large{\bf 9.4 Refractive Optics: Lenses}\\\\
$a$ = \vbox{\hrule width 4cm} ({\bf 1 point})\\\\
$b$ = \vbox{\hrule width 4cm} ({\bf 1 point})\\\\
$f$ = \vbox{\hrule width 4cm} ({\bf 2 points}) --- focal length of the small positive lens \\\\

Can you find a focus with the small negative lens? What appears to be happening? ({\bf 4 points})
\vspace{4cm}


Draw how light behaves when encountering two types of lenses on the attached worksheet (Fig. 9.5).\\\\
How does the behavior of these two lenses compare with the behavior of mirrors? Note some similarities between what you drew for the mirrors (Fig. 9.4) and the lenses (Fig. 9.5). ({\bf 5 points})
\vspace{6cm}

The focal length of the large lens is $F$ = \vbox{\hrule width 4cm} cm ({\bf 2 points})\\\\

\newpage
{\bf Making a Telescope}\\

\textit{Galileo Telescope}\\\\
The distance between the two lenses is $N$ = \vbox{\hrule width 4cm} cm ({\bf 2 points})\\\\

Describe what you see through the Galileo telescope. What does the image look like? Is it distorted? Are there strange colors? What is the smallest set of letters you can read? Is the image right-side up? Any other interesting observations? ({\bf 5 points})
\vspace{4cm}

\textit{Kepler Telescope}\\\\
The distance between the two lenses is $P$ = \vbox{\hrule width 4cm} ({\bf 2 points})\\\\

Describe what you see through the Kepler telescope. What does the image look like? Is it distorted? Are there strange colors? What is the smallest set of letters you can read? Is the image right-side up? Any other interesting observations? ({\bf 5 points})
\vspace{4cm}

Compare the two telescopes. Which is better? What makes it better? Why was Kepler's version not popular until many years later? ({\bf 5 points})
\vspace{4cm}

\newpage
{\bf The Magnifying and Light Collecting Power of a Telescope}\\\\

The magnification of the Kepler telescope is $M$ = \vbox{\hrule width 4cm} times. ({\bf 1 point})\\\\

Find $f$ for the small negative lens eyepiece of Galileo telescope that you built. Use that to determine the magnification $M$ of this telescope. (\textit{Hint: look at Fig. 9.3 and use the fact that $N = F - f$.})\\\\
$f$ = \vbox{\hrule width 4cm} ({\bf 2 points})\\\\\\

$M$ = \vbox{\hrule width 4cm} ({\bf 1 point})\\\\\\

Compare the magnifications of the Kepler and Galileo telescopes. (\textit{Hint: take the ratio by dividing. This will tell you how much more powerful one is than the other.}) ({\bf 2 points})
\vspace{4cm}

What do you think of the quality of the images that these simple telescopes produce? Amazingly enough, the simple telescopes you constructed today are much better than what Galileo used!



\end{document}
