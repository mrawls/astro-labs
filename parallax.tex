\documentclass[notitlepage]{article}
\newcounter{qcounter}
\usepackage{graphicx}
\usepackage{array}
\usepackage{booktabs}
\usepackage{fancyhdr}
%\pagestyle{empty}
%\pagestyle{fancy}
%\fancyhf{}
%\fancyfoot[HR]{Page \thepage\ }
%\renewcommand{\headrulewidth}{0pt}
%\renewcommand{\footrulewidth}{0pt}
\textwidth	7.00in
\textheight	8.50in
\oddsidemargin	-0.35in
\topmargin	-0.5in
\parindent	0.00in
%\newcolumntype{m}{{\centering\arraybackslash}m{\dimexpr.25\linewidth-2\tabcolsep}}

\begin{document}

\begin{center}
\Large{\bf Lab 8:}
\Large{\bf Measuring Distances Using Parallax}\\
\vspace{0.5cm}
Names: \vbox{\hrule width 10cm}
\end{center}
%\vspace{1cm}

%\begin{list}{\bf Exercise \arabic{qcounter}:~}{\usecounter{qcounter}}

\large{\bf 8.1.1 Parallax in the classroom}\\\\
\textit{Measure each object's shift THREE TIMES, and then take the average. \\Try to measure to the \textbf{nearest fraction of a tick mark}.}
\begin{itemize}
\item How many tick marks did the object move at the closest distance?  ({\bf 2 points})\\\\
\item How many tick marks did the object move at the middle distance?  ({\bf 2 points})\\\\
\item How many tick marks did the object move at the farthest distance?  ({\bf 2 points})\\\\

\item What is your estimate of the {\it uncertainty} in your measurements of the apparent motion? In other words, how many tick marks could you have been off by? ({\bf 2 points})\\
\vspace{1.5cm}

\item Qualitatively, what do you see? As the object gets farther away, is the apparent motion smaller or larger? \\
\vspace{1.5cm}

\item How many tick marks did the object move from the more widely separated vantage points? (Move your head over about a foot using just one eye instead of blinking your eyes)\\

\item For an object at a fixed distance, how does the apparent motion change as you observe from a more widely separated vantage points? \\
\vspace{3cm}
\end{itemize}



\large{\bf 8.1.2 Measuring distances using parallax}

\begin{itemize}
\item Number of degrees for the entire background ruler: \vbox{\hrule width 4cm} \\

\item Number of tick marks in the whole ruler: \vbox{\hrule width 4cm} \\

\item Number of degrees in each tick mark: \vbox{\hrule width 4cm} \\

\item How many degrees did the object appear to move at the closest distance? ({\bf 2 points})\\\\

\item How many degrees did the object appear to move at the middle distance? ({\bf 2 points})\\\\

\item How many degrees did the object appear to move at the farthest distance? ({\bf 2 points})\\\\

\item Based on your estimate of the uncertainty in the number of tick marks each object moved, what is your estimate of the uncertainty in the number of degrees that each object moved? (\textit{Hint: change one of your tick mark measurements by adding the uncertainty. Then re-calculate the angle.}) ({\bf 2 points})
\vspace{3cm}

\item What is the distance between your eyes? \vbox{\hrule width 4cm}  ({\bf 2 points})\\
\end{itemize}

\large{\bf The ``Non-Tangent'' way to figure out distances from angles}\\
\begin{itemize}
\item Distance when object was at closest distance: \vbox{\hrule width 4cm} ({\bf 2 points})\\

\item Distance when object was at middle distance: \vbox{\hrule width 4cm} ({\bf 2 points})\\

\item Distance when object was at farthest distance: \vbox{\hrule width 4cm} ({\bf 2 points})\\

\item Estimate the uncertainty in your measurements of the distances to the objects. (\textit{Hint: change one of your angle measurements by adding the uncertainty. Then re-calculate the distance.}) ({\bf 2 points})\\
\vspace{3cm}\\

\item Now go and measure the \textit{actual} distances to the locations of the objects. How well did the parallax technique work? Are the differences between the actual measurements and your parallax measurements within your estimated errors? If not, can you think of any reasons why your measurements might have some additional error in them? \textit{Be specific!} ({\bf 5 points})\\
\vspace{4cm}


\end{itemize}


\large{\bf 8.1.3 Using Parallax to measure distances on Earth, and within the Solar System}\\

\begin{itemize}

\item Using the small angle formula, and your eyes, what would be the parallax angle (in degrees) for Organ Summit, the highest peak in the Organ mountains, if the Organ Summit is located 12 miles (or 20 km) from this classroom? ({\bf 2 points})\\
\vspace{4.5cm}

\newpage

\item Calculate the parallax angle for Mars using a baseline of 1000 km. ({\bf 2 points})\\
\vspace{4.5cm}

\end{itemize}

\large{\bf 8.1.4 Distances to stars using parallax, and the ``Parsec''}\\
\begin{itemize}
\item If a star has a parallax angle of $\theta$ = 0.25'', what is its distance in Parsecs? ({\bf 1 point})\\\\

\item If a star is at a distance of 5 Parsecs, what is its parallax angle? ({\bf 1 point})\\\\

\item If a star is at a distance of 5 Parsecs, how many light years away is it? ({\bf 1 point)}\\\\

\end{itemize}







\large{\bf 8.2 Questions}

\begin{enumerate}

\item How does the parallax angle change as an object is moved further away? Given that you can usually only measure an angular motion to some accuracy, would it be easier to measure the distance to a nearby star or a more distant star? Why? ({\bf 4 points})\\
\vspace{2.5cm}

\newpage

\item Relate the experiment you did in lab to the way parallax is used to measure the distances to nearby stars in astronomy. Describe the process an astronomer has to go through in order to determine the distance to a star using the parallax method. What do your two eyes represent in that experiment? ({\bf 5 points})\\
\vspace{4cm}

\item How far away is Star P? ({\bf 11 points})
\begin{itemize}
\item Measure the distance between Stars A and B on paper and create a map scale
\item Measure how far Star P moves on paper (measured cm)
\item Convert this to an angular shift (arcseconds)
\item Use the parallax equation to solve for distance (parsecs)
\item \textit{Remember, the apparent angular shift is $\alpha$ and the parallax angle is $\theta = \alpha/2$}
\end{itemize}

\vspace{9cm}

\newpage

\item Imagine that you did the classroom experiment by putting your partner all the way against the far wall. How big would the apparent motion be relative to the tick marks? What would you infer about the distance to your partner? Why do you think this estimate is incorrect? What can you infer about where the background objects in a parallax experiment need to be located? ({\bf 7 points})\\
\vspace{6cm}

\end{enumerate}

\end{document}

